\documentclass{beamer}

\usepackage[utf8]{inputenc}
\usepackage{lmodern}
\usepackage{xcolor}
\usepackage{graphicx}
\usepackage{tikz}

\title{Introduction to (zk-)SNARKs}
\author{Daniel Barrero}
\institute{FLAGlab\\ Universidad de los Andes}


\begin{document}
\frame{\titlepage}

\begin{frame}{Proofs of knowledge}
	\begin{itemize}
		\item Two characters: $\mathcal{V}$ (Verifier, computationally weak)
			and $\mathcal{P}$ (Prover, computationally strong).
		\item They exchange information about an NP relation $R \subset
			X \times W$ of $($\emph{statement, witness}$)$ pairs.
		\item Given a statement $x$, $\mathcal{P}$ must
			``convince''/\emph{prove} to $\mathcal{V}$ that she knows
			a witness $w$ such that $(x,w) \in R$ (Hash functions,
			graph problems\ldots).
	\end{itemize}
\end{frame}

\begin{frame}{SNARKs}
	For a knowledge proof system $(\mathcal{K}, \mathcal{P}, \mathcal{V})$ to be
	a SNARK it must satisfy:
	\begin{itemize}
		\item \emph{Succinctness}: Proof must be small and verification
			must be fast. Formally,
			\[
				\begin{array}{c}
					|\pi| \in O(\log |T_R|)\\
					\tau_{\mathcal{V}}(x, \pi) \in O(\log |T_R|)
				\end{array}
			\]
			where $T_R$ denotes a Turing machine that verifies the
			relation $R$.
		\item \emph{Non-interactiveness}: After $\mathcal{P}$ and $\mathcal{V}$
			are both given the statement $x$, $\mathcal{P}$ sends his
			proof $\pi$ without receiving any input from $\mathcal{V}$,
			and verification also doesn't require any interaction with
			the prover.
	\end{itemize}
\end{frame}

\begin{frame}{Completeness, knowledge soundness}
	\begin{itemize}
		\item Completeness:
		\item Soundness: for every adversary. extractor.
	\end{itemize}

\end{document}
