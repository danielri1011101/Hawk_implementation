\documentclass{beamer}

\usepackage[utf8]{inputenc}
\usepackage{lmodern}
\usepackage{xcolor}
\usepackage{graphicx}
\usepackage{tikz}

\title{Introduction to (zk-)SNARKs}
\author{Daniel Barrero}
\institute{FLAGlab\\ Universidad de los Andes}


\begin{document}
\maketitle

\section{January 14 -- ?}
I was about to learn (in bird's eye view) the working of the \texttt{verify}
subroutine.

However, I started reviewing the norms and inner products.

\textbf{Question.} Are inner products closed under scalar multiplication? Yes,
if and only if the scalar is positive.

\section{NTRUsolve}
The NTRU equation is
\[
	fG - gF = q \mod X^n + 1
\]
where $f$ and $g$ are given, and the idea is to solve for $F$ and $G$. All the
polynomials in the equation are in
\(
	\mathbb{Z}[X]/\encl{X^n + 1}.
\)

\subsection{Resultants} The resultant of two polynomials $p$ and $q$ is zero if and
only if they have a common root. Thus one way to define it is
\[
	\mathrm{Res}(p,q)= p_n^m\prod_{i=0}^{n-1}q(\gamma_i)
\]
where the $\gamma_i$ are the roots of $p$ and $m$ is the degree of $q$.

Given a monic polynomial
\(
	\phi \in \mathbb{C}[X]
\)
of degree $n$ and a polynomial
\(
	p \in \mathbb{C}[X]/\encl{\phi}
\)
we denote by $\rema{p}{\phi}$ the matrix whose $j-$th row consists of the
coefficients of
\(
	X^{j-1}p \mod \phi,
\)
as in
\[
	\rema{p}{\phi} =
	\left[
		\begin{array}{c}
			p \mod \phi\\
			Xp \mod \phi\\
			\vdots\\
			X^{n-1}p \mod \phi
		\end{array}
	\right]
\]

\bigskip
Recall that if
\(
	\vec{x}= x_0, x_1, \ldots, x_{m-1}
\)

is a sequence of numbers, its $n-$th \emph{Vandermonde matrix} is given by

\[
	V(\vec{x},n)^i_j= x_i^j.
\]

In particular, given a polynomial $\phi$ with $n$ distinct roots over $\mathbb{C}$,
we defie $V_\phi$ as the $n-$th Vandermonde matrix of its roots:

\[
	V_\phi=
	\left[
		\begin{array}{ccccc}
			1 & \alpha_0 & \alpha_0^2 & \ldots & \alpha_0^{n-1}\\
			1 & \alpha_1 & \alpha_1^2 & \ldots & \alpha_1^{n-1}\\
			\vdots & \vdots & \ddots & & \vdots \\
			 1 & \alpha_{n-1} & \alpha_{n-1}^2 & \ldots & \alpha_{n-1}^{n-1}\\
		\end{array}
	\right]
\]

\bigskip
\begin{thm}
	If the polynomial $g$ has $n$ distinct roots over $\mathbb{C}$ and $f$ has
	degree less than $n$, then
	\[
		\mathrm{Res}(g,f) = \det \rema{f}{g}.
	\]
\end{thm}

\begin{proof}
	Let $D$ denote the diagonal matrix whose entries are the evaluations of $f$
	at the roots of $g$. Then upon inspection it is easily seen that
	\[
			V_g (\rema{f}{g})^{\textsf{t}} = D V_g,
	\]
	and therefore
	\[
		(\rema{f}{g})^{\textsf{t}} = V_g^{-1} D V_g.
	\]
	The result then follows from the fact that
	\(
		\det\rema{f}{g} = \det(\rema{f}{g})^{\textsf{t}}
	\)
	and from the multiplicativity of the determinant.
\end{proof}
\end{document}
